
% SLIDE GENERATION INSTRUCTIONS

% For each slide I will paste in latex code with comments.
% For each comment, identify how to best address it
% considering the other comments and the context from the paper.

% To address each comment, ONLY write the changes you propose
% in code blocks I can easily paste into the latex document.
% For each code block explain HOW your changes address my comments.
% Think step-by-step and always identify your previous issues first.
% Make sure to output latex that is at most 80 chars per line,
% make plenty line breaks to ensure good readability.

% When writing slides, please make sure to keep this logical
% structure so that it is still understandable.
% It must have a narrative, that is very important.
% Split one 'slide' into multiple pages using the package and
% \only<slide number, e.g. 1, 2 or even '1-2' for both> {..} to make it more easy to follow
% for listeners.

% Your output should look like this:
% ## Slide <num>: <slide title>
% This slide addresses the comment <comment from latex> by
% <one sentence explanation of slide contents>.
% ```latex
% \only<1> { % Slide 1 content, narrative, etc. }
% \only<2-3> { % Slide 2, transition to 3 }
% \only<3> { % Slide 3 content, narrative, etc. }
% ```

\documentclass{beamer}
\usetheme{Madrid} 
\usepackage{graphicx}
\usepackage{hyperref}
\usepackage{tikz}
\usepackage{pgfplots}
\usepackage{amssymb}

% See https://tex.stackexchange.com/a/5255
\usepackage{amsmath}
\DeclareMathOperator*{\argmax}{arg\,max}
\DeclareMathOperator*{\argmin}{arg\,min}

\usetikzlibrary{positioning}

% Center the frame title and remove the footer
% \setbeamertemplate{frametitle}[default][center]
\setbeamertemplate{footline}[page number]{}

% Define some math operators
\newcommand{\E}{\mathbb{E}}

% Uncomment this to remove the navigation symbols
% \setbeamertemplate{navigation symbols}{}

% Includes names, dates, etc.
% This way, they are not used in the LLM when generating slides.
\title{Intro.M.L. Group Project Paper}
\author{Simin.tahmasebi.gandomkari; chujie.wang; winnie.tan; eric.tubbs}
\date{06/30/2025}
%\section{Introduction}

% Frame 1: Motivation & Research Question
\begin{frame}{Motivation \& Core Research Question}
  \begin{itemize}
    \item tk
  \end{itemize}
\end{frame}


\begin{document}

\begin{frame}
    \titlepage
\end{frame}

\begin{frame}{Outline}
    \tableofcontents
\end{frame}

\section{Introduction}

% Frame 1: Motivation & Research Question
\begin{frame}{Motivation \& Core Research Question}
  \begin{itemize}
    \item tk
  \end{itemize}
\end{frame}
\section{Data Cleaning}

% Frame 1: Motivation & Research Question
\begin{frame}{Motivation \& Core Research Question}
  \begin{itemize}
    \item tk
  \end{itemize}
\end{frame}
\section{Explorative Data Analysis (EDA)}

% Frame 1: Motivation & Research Question
\begin{frame}{Motivation \& Core Research Question}
  \begin{itemize}
    \item tk
  \end{itemize}
\end{frame}
\section{Model Fitting}

%MODEL FITTING NOTES
%what is model fitting?

%Model Fitting is a measurement of how well a machine learning model adapts
% to data that is similar to the data on which it was trained. The fitting process is 
%generally built-in to models and is automatic.

%So this is basically the same as the backpropagation that we do in our python file, with the neural network
%wit hthe neural network being able to backpropagate its changes throguh the rest of the model
%This makes life relatively easy for us, no?

% Frame 1: Model Fitting

Model fitting is the process by which a machine learning model adjusts its internal 
parameters to minimize the discrepancy between its predictions and the observed data. In supervised learning, we define a loss function
 \(L(\theta)\) that measures this error—where \(\theta\) denotes all trainable parameters—and then optimize:

\[
\theta \leftarrow \theta - \eta\,\nabla_{\theta}L(\theta),
\]

using gradient‐based methods such as (stochastic) gradient descent. 

Within neural networks, backpropagation efficiently computes \(\nabla_{\theta}L\) by applying the chain rule through each layer. 
Iterating this update over many epochs (and possibly minibatches) lets the model “fit” patterns in the training set. 
Although modern libraries automate these steps, understanding loss optimization and backpropagation is essential for diagnosing convergence issues, tuning hyperparameters (learning rate, batch size, etc.), and avoiding overfitting.

% If you’re in Beamer, drop the above paragraphs and use this frame instead:
\begin{frame}{Model Fitting}
  \begin{itemize}
    \item Definition: tuning model parameters to reduce prediction error
    \item Loss function \(L(\theta)\) optimized via gradient descent
    \item Backpropagation computes gradients layer by layer
    \item Key hyperparameters: learning rate, batch size, epochs
    \item Understanding fitting helps with convergence diagnostics and generalization
  \end{itemize}
\end{frame}

\section{Model Diagnostics}

Once the network is trained, it’s crucial to verify that it learned meaningful patterns rather than memorizing noise. In this section we cover loss‐curve analysis, hyperparameter validation, error‐propagation checks and final performance metrics.

\subsection{Loss‐Curve Analysis}

We plot the training loss over iterations to ensure smooth decay and detect plateaus or oscillations:



\[
L^{(t)} = \frac{1}{2N}\sum_{i=1}^{N}\bigl(y_i - \hat y_i^{(t)}\bigr)^2,
\]


where \(\hat y_i^{(t)}\) is the network’s output at iteration \(t\). A monotonically decreasing loss curve indicates stable learning; sudden spikes or flat regions suggest learning‐rate issues or vanishing gradients.

\begin{verbatim}
plt.plot(clf.loss_curve_)
plt.title{Loss Curve}
plt.xlabel{Iterations}
plt.ylabel{Cost}
plt.show()
\end{verbatim}

\subsection{Hyperparameter Search}

GridSearchCV systematically explores combinations of layer sizes, activation functions, solvers and regularization strengths. We examine cross‐validation accuracy and standard deviation to choose a model that generalizes well:

\begin{itemize}
  \item \texttt{hidden\_layer\_sizes}: (150,100,50), (120,80,40), (100,50,30)
  \item \texttt{activation}: logistic vs.\ relu
  \item \texttt{solver}: sgd vs.\ adam
  \item \texttt{alpha}: 1e–4, 5e–2, 1
  \item learning‐rate schedules and initial rates
\end{itemize}

After fitting:
\begin{verbatim}
print(grid.best_params_)
print("Accuracy: {:.2f}".format(
    accuracy_score(y_test, grid.predict(X_test))
))
\end{verbatim}

\subsection{Custom Backpropagation Checks}

To validate that gradients and weight updates behave correctly, we re‐implement a minimal two‐layer MLP.  At each sample we compute:



\[
\delta^2 = (y - a^2)\,\sigma'(z^2), 
\qquad
\delta^1 = \bigl(W^2\,\delta^2\bigr)\circ\sigma'(z^1),
\]


and update  
\(\;W \gets W + \eta\,a\,\delta\),  
\(\;b \gets b + \eta\,\delta\).  
Convergence is signaled when the epoch loss falls below a small tolerance.

\subsection{Final Performance Metrics}

Beyond accuracy, inspect:
\begin{itemize}
  \item Confusion matrix (precision, recall, F1‐score)
  \item Receiver‐Operating Characteristic (ROC) curve and AUC
  \item Calibration plots for predicted probabilities
  \item Profit or cost‐benefit analysis if thresholds carry economic impact
\end{itemize}

---

% If you’re using Beamer, summarise on one slide:
\begin{itemize}
  \item Plot loss curve → learning dynamics
  \item GridSearchCV → robust hyperparameters
  \item Backprop sanity‐check → gradient & weight updates
  \item Metrics: accuracy, confusion matrix, ROC/AUC, calibration
  \item Economic/profit evaluation via custom thresholds
\end{itemize}


%diagonostics notes:
%   >needed to recode the revenue var since it was in float
%   >had to normalize the revenue var since it was funky
%   >needed to log the revenue var since it was funky, particularly right skewed
%   >had to recode the green building var since it had some "missing values" that were
%   hidden as strings
%   >needed to drop the vars that were ugly values & the nonsense
\section{Plotting \& Discussion}

% Frame 1: Motivation & Research Question
\begin{frame}{Motivation \& Core Research Question}
  \begin{itemize}
    \item tk
  \end{itemize}
\end{frame}
\include{chapters/con}

\end{document}


%//PAPER
%    1. INTRO (SIMIN)
%    2. DATA CLEANING (SIMIN)
%    3. E.D.A. (WINNIE)
%    4. MODEL FITTING (ERIC)
%    5. DIAGNOGSTICS (ERIC)
%    6. PLOTTING & DISCUSSION (CHUJIE)
%    7. CONCLUSION